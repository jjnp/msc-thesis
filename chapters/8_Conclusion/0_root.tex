% target: 4ish pages

% Philipps DA: 1p intro, 1p per RQ, 1p future work
\chapter{Conclusion}
Edge computing has been proposed as a new computing paradigm to address the computational needs of new applications such as real-time cognitive assistance, large scale analytics from \gls{iot} sensors, or deep learning based image analysis on resource constrained devices.
Edge computing brings with it a number of significant implementation challenges.
In particular, edge computing environments are heterogeneous in hardware, software and networking conditions, featuring drastically different hardware, various operating systems, and a wide range of network conditions.
In addition to the system itself being heterogeneous, the surrounding conditions can be highly dynamic with request rates, clients, running applications, and the very structure of the system changing over time, in some scenarios even very quickly.

Currently, this complexity needs to be resolved by system developers themselves, which is not only very difficult, but is typically implementation-specific and thus not efficient in the long run.
Serverless computing has been proposed as an abstraction layer on top of edge computing systems with the goal of alleviating the need to deal with this complexity for developers.
Instead, the serverless edge computing framework is supposed to handle the tasks of scheduling and scaling functions, as well as routing them in such a way that the requirements of the application are fulfilled.
While serverless systems have been adapted to be edge-capable, there are some challenges that remain.
Network bound workloads in particular are an area where the performance of serverless edge computing systems was not in line with expectations.

This is the area of serverless edge computing this work aims to improve upon.
We perform an initial analysis of the load balancing component of the serverless system, which we suspect is the key component that needs to be adapted for network bound workloads to perform better.
Based on these results we propose to improve load balancing for serverless edge computing through two adaptations.
First, our approach changes the operating logic of the load balancer itself, moving away from the simple round robin load balancing current systems default to, towards a weighted response time implementation that tunes and adapts weights dynamically based on a least response time logic.
Second, we propose an osmotic pressure based approach to scaling and scheduling load balancer replicas, giving load balancers their own scaling and scheduling logic separate from the one regular serverless functions use, which allows the load balancer scheduling to consider the location of both function replicas and clients.

Results show that our approach not only significantly improves response times of network bound workloads, but that it also opens up future opportunities for sophisticated deployment scenarios that have more differentiated performance requirements.

\section{Research Questions}

\subsubsection{How can current scaling, placement, and routing techniques for load balancers be changed, such that the overall performance of the serverless edge computing system improves?}

In their current state serverless edge computing systems treat load balancers as "just another function", not paying special attention to how they are scheduled and scaled.
Likewise the routing (i.e. load balancing) decisions themselves are made by algorithms designed for and tested in cloud-centric environments.
To improve performance, both of these areas have to be adapted to deal with the challenges of the edge computing environment.\\
To improve routing decisions we move away from simple round robin load balancing, which is incapable of addressing the heterogeneous device capabilities found in edge computing.
Instead we use a variant of weighted round robin load balancing, which is better suited to deal with the differently performant devices and accordingly differently performant upstreams.
Because the performance of different devices is not known beforehand, and can change dynamically at runtime we use the observed response time of past requests as a black box metric and stand in for the expected performance of the upstream in question.
Based on a moving average of these past response times and a least response time logic we assign each of the upstreams their respective weights used by weighted round robin.
This evaluation of observed response times and weight mappings happens continuously and as a result it can adapt to changing system conditions as well.
For scaling and scheduling we introduce an approach inspired by the concept of osmotic pressure we call osmotic scaling and scheduling.\\
The load balancer scaling and scheduling component needs to take into account where in the network client requests are originating from, where function replicas to handle those requests are located, and where load balancer instances are already present.
To this end we introduce the notion of osmotic pressure.
For each node that does not already have a load balancer deployed, we calculate a pressure metric that depends on how many requests it would receive if there was a load balancer present, how close the clients sending these requests are, and how close the function replicas required to process the requests are located.
If the pressure exceeds a set threshold a new load balancer replica is added on that node.
Likewise a negative pressure is calculated for nodes which are already host to a load balancer instance, where requests that would be more efficiently handled by other running load balancer instances contribute to negative pressure.
If the pressure falls below the downscaling pressure threshold the load balancer replica is removed.\\
Through the introduction of these techniques for load balancing, scaling and scheduling the overall performance of the system improves, particularly for network bound workloads.


\subsubsection{How much of a performance improvement can be gained from optimizing the scaling,
placement and routing decisions of load balancers in serverless edge computing
systems?}

Our initial evaluations show that the diagnosis of load balancer location and load balancing technique was correct.
In the first evaluation, where we tested the new load balancing implementation and compared a scenario with a centralized round robin load balancer on one hand, and our adapted version running on every node on the other, the mean total response time could be improved by between 81\% and 606\%, depending on the scenario.
These performance gains hold, albeit in a less dramatic way in our more stringent and realistic evaluation of the osmotic scaling and scheduling component.
In these tests the results still show an improvement between in mean \gls{trt} between 43\% and 229\% compared to a non-centralized round robin implementation, which corresponds to a reduction of -30\% to -69\% in \gls{trt}.
The improvements are larger when looking not only at the mean values, but at the different percentiles as well.\\
The higher the percentile, the bigger the \gls{trt} improvement generally is.
What this means is that our approach not only improves the performance of the system on average, it also reduces the variance, making the performance gains more reliable.
The performance gains are realized by having decreased network transfer times between client and load balancer, and load balancer and function replica.
This is expected and optimizing this part of the response time is the primary implementation goal of the osmotic scaling and scheduling approach we present.
In addition to improving network transfer times, our approach realizes part of its performance gains via improvements \gls{fet}, which are a result of the load balancing implementation relying on a black box metric that also encompasses \gls{fet} and thus partially optimizing for it.

\subsubsection{How do edge optimized scaling and placement techniques for load balancers, including the load balancing techniques themselves, affect the overall system behaviour
and characteristics in regard to their key performance metrics?}

Our approach also has effects on the system beyond its improvements in \gls{trt}.
Notably it positions load balancers efficiently, achieving performance that would require the deployment of more load balancer instances when using current scheduling methods.
This is particularly relevant as our experiments with real hardware show that depending on the device and operating system used memory and CPU consumption can vary significantly.\\
In addition we can observe that our approach slightly reduces the number of function replicas deployed in the serverless system, when using the default scaling methodology of OpenFaaS\cite{openfaas} as a representative example of serverless frameworks in general.
While these reductions in function scale are too small to be considered a stable feature of our approach, it indicates that it can likely be integrated into serverless frameworks without adverse effect on function scale.\\
As one would expect based on the faster network transfer times, our approach reduces the share of requests that get routed to far-away function replicas.
In an evaluation with three distinct cities far apart from each other our approach routes only 1.9\% of requests to a function replica outside the city, compared to the 65.2\% of round robin based load balancing currently used in serverless frameworks.
\section{Future Work}

% future work idea: it is relatively obvious that there is a non-linear relationship between load balancer scale and the performance limit they provide. The question now is how the linearity and scales here are influenced by the network itself. Can there, based on the network structure, be a way to determine the "optimal" number, or predict the impact sufficiently accurately that one can derive the "correct" scale given certain SLA requirements?

% how does information sharing between load balancerns contribute to their performance, and how can this be solved on an engineering-level to avoid exponential growth in data transfer sizes? also: how would the stochastic congestion game idea factor into this?

Our work provides a basis on which future work can be built, both to address limitations of our approach, as well as to further the capabilities of serverless edge computing.
This section provides a list of some of the most important areas for future work.
\begin{itemize}
    \item A continued evaluation of our approach in different deployment scenarios and network topologies could give further insight into its advantages and limitations. Based on this our approach could be changed to bring it closer to a state where it can reliably be used in real-world deployments.
    \item Building on our results, that show how response time percentiles and general service levels are influenced by load balancer scaling and implementation, a more sophisticated approach could be developed that takes into account and optimizes for different \gls{qos} levels on a per-function basis.
    \item In keeping with the previous point about different \gls{qos} levels, our approach could be modified to explicitly model the requirements of functions and then automatically tune its parameters accordingly. This would stand in contrast to current, statically configured implementations. In addition the need to do so is indicated by our experimental results, which show that for different environments different configurations are required to achieve optimal performance.
    \item Joint scheduling and scaling of functions and load balancers offers the potential for additional performance gains, where the information about required resources gathered by the load balancer could help make more informed decisions about function replica scale and location.
    \item Lastly, the topic of standardized edge computing scenarios for benchmarking remains a worthwhile area to improve upon. While there is effort to develop systematic benchmarks for edge computing, for example EdgeBench\cite{edgebench}, there is much to be done in terms of modelling network topologies and dynamically changing environments. Such a suite of standardized benchmarks and deployment scenarios would help to make competing approaches more directly comparable, and results more transparent.
    
\end{itemize}   