% target: 4ish pages

% Philipps DA: 1p intro, 1p per RQ, 1p future work
\chapter{Conclusion}
Edge computing has been proposed as a new computing paradigm to address the computational needs of new applications such as real-time cognitive assistance, large scale analytics from \gls{iot} sensors, or deep learning based image analysis on resource constrained devices.
Edge computing brings with it a number of significant implementation challenges.
In particular, edge computing environments are heterogeneous in hardware, software and networking conditions, featuring drastically different hardware, various operating systems, and a wide range of network conditions.
In addition to the system itself being heterogeneous, the surrounding conditions can be highly dynamic with request rates, clients, running applications, and the very structure of the system changing over time, in some scenarios even very quickly.

Currently, this complexity needs to be resolved by system developers themselves, which is not only very difficult, but is typically implementation-specific and thus not efficient in the long run.
Serverless computing has been proposed as an abstraction layer on top of edge computing systems with the goal of alleviating the need to deal with this complexity for developers.
Instead, the serverless edge computing framework is supposed to handle the tasks of scheduling and scaling functions, as well as routing them in such a way that the requirements of the application are fulfilled.
While serverless systems have been adapted to be edge-capable, there are some challenges that remain.
Network bound workloads in particular are an area where the performance of serverless edge computing systems was not in line with expectations.

This is the area of serverless edge computing this work aims to improve upon.
We perform an initial analysis of the load balancing component of the serverless system, which we suspect is the key component that needs to be adapted for network bound workloads to perform better.
Based on these results we propose to improve load balancing for serverless edge computing through two adaptations.
First, our approach changes the operating logic of the load balancer itself, moving away from the simple round robin load balancing current systems default to, towards a weighted response time implementation that tunes and adapts weights dynamically based on a least response time logic.
Second, we propose an osmotic pressure based approach to scaling and scheduling load balancer replicas, giving load balancers their own scaling and scheduling logic separate from the one regular serverless functions use, which allows the load balancer scheduling to consider the location of both function replicas and clients.

Results show that our approach not only significantly improves response times of network bound workloads, but that it also opens up future opportunities for sophisticated deployment scenarios that have more differentiated performance requirements.

\section{Research Questions}
\section{Future Work}

% future work idea: it is relatively obvious that there is a non-linear relationship between load balancer scale and the performance limit they provide. The question now is how the linearity and scales here are influenced by the network itself. Can there, based on the network structure, be a way to determine the "optimal" number, or predict the impact sufficiently accurately that one can derive the "correct" scale given certain SLA requirements?

% how does information sharing between load balancerns contribute to their performance, and how can this be solved on an engineering-level to avoid exponential growth in data transfer sizes? also: how would the stochastic congestion game idea factor into this?

Our work provides a basis on which future work can be built, both to address limitations of our approach, as well as to further the capabilities of serverless edge computing.
This section provides a list of some of the most important areas for future work.
\begin{itemize}
    \item A continued evaluation of our approach in different deployment scenarios and network topologies could give further insight into its advantages and limitations. Based on this our approach could be changed to bring it closer to a state where it can reliably be used in real-world deployments.
    \item Building on our results, that show how response time percentiles and general service levels are influenced by load balancer scaling and implementation, a more sophisticated approach could be developed that takes into account and optimizes for different \gls{qos} levels on a per-function basis.
    \item In keeping with the previous point about different \gls{qos} levels, our approach could be modified to explicitly model the requirements of functions and then automatically tune its parameters accordingly. This would stand in contrast to current, statically configured implementations. In addition the need to do so is indicated by our experimental results, which show that for different environments different configurations are required to achieve optimal performance.
    \item Joint scheduling and scaling of functions and load balancers offers the potential for additional performance gains, where the information about required resources gathered by the load balancer could help make more informed decisions about function replica scale and location.
    \item Lastly, the topic of standardized edge computing scenarios for benchmarking remains a worthwhile area to improve upon. While there is effort to develop systematic benchmarks for edge computing, for example EdgeBench\cite{edgebench}, there is much to be done in terms of modelling network topologies and dynamically changing environments. Such a suite of standardized benchmarks and deployment scenarios would help to make competing approaches more directly comparable, and results more transparent.
    
\end{itemize}   