% 10ish pages, about 8 in text, about 2 in figures
% Results had 18 pages, 50/50 ish split
% probably some 20ish pages here
\chapter{Evaluation}

% Stuff that goes here:
% Faas-sim: what is it, how does it work, what are its models, how does it deal with containers/kubernetes/openfaas/etc. What is simulated, what isn't? How are network flows simulated?
% Request pattern stuff.
% what types devices are there
% what functions are there, how do they perform on each device?
% image sizes, pulls, request payload sizes, all that stuff
% topologies: What are there and how are they built? real measurements from papers, etc. etc.
% -> city level, both old and modern, visualizations, etc.

% practical: how did we eval traefik
% real cluster, galileo (briefly), what devices are there? Auto-responder type app, payload size, response time, etc.

% Experiments:
% 1) Prelims
% 2) traefik resource eval
% 3) more realistic topo
% 4) load balancer count tests
% 5) LB hyperparameter tests
% 6) the osmotic stuffs?

This chapter describes the experiments we conduct as well as their results.
First, we described the initial evaluation we performed which significantly informed our exact approach.
It provides first insights into the performance improvement one can expect, and also helps uncover potentially unexpected system behaviour.
After this initial set of experiments we continue our evaluation with the implementation and parameter configuration of the load balancers.
From there we continue with experiments related to the effect and cost load balancers have.
As we described, our serverless function simulator uses real-world values whenever possible to inform its simulation.
To this end we test the resources used by a single load balancer instance under various conditions on different kinds of hardware.
Because the overall resource cost load balancers incur is a function of the resource consumption of each instance and the number of instances deployed, we next evaluate load balancer scale.
Lastly we evaluate our osmotic scaling and scheduling approach.
We start this evaluation by testing how our osmotic scaling behaves under more realistic system conditions, as well as how it affects the serverless system.
Next, we test the effect of different pressure thresholds on system performance.
Combined with previous experiments this informs how scaling parameters result in different levels of performance and load balancer resource consumption.
The last experiment conducted tests how our osmotic approach handles dynamically changing system conditions, exploring how the scaler and scheduler behave when requests change their origin within the system over time.

\section{Initial Assessment}
% got 5,5-6 pages
This initial assessment is the first experiment we performed in the course of this work.
We performed it very early on to get an initial impression on whether our diagnosis of the problem, namely that load balancers are making ineffective decisions and are themselves located too far from clients, is accurate.
We also hoped to get a first impression of how large of a performance uplift might be achievable, and thus whether the performance improvement would justify the additional complexity our approach adds to the system.

The overarching question we want to answer with these experiments is whether more complex load balancing, such as least response time load balancing, and moving the load balancers closer to the clients leads to overall performance improvements.
We also want to understand what impact on performance we could expect from implementing only one of the two proposed improvements.

\subsubsection{Setup}
To answer these questions we test four load balancer configurations in three different scenarios.
\paragraph{Load Balancer Setup}
To assess the role of the load balancer implementation, we compare typical round robin load balancing, as it is found in current serverless frameworks such as OpenFaaS\cite{openfaas} and their underlying container orchestration service\cite{kubernetes}, and least response time load balancing to represent more sophisticated load balancing decisions.

Since this experiment is performed before the others we rely on an initial parametrization and implementation, which differs slightly from the one we ultimately propose.
The weight range is [1;10], with a scaling factor of 1, and a weight update interval of 15 seconds.
Furthermore we rely on a different implementation of weighted round robin\cite{wrr-kblinux}, which is functionally very similar but works through upstreams iteratively by their weight, instead of the more intermixed upstream selection we describe in our approach.
In this implementation, there is also a notion of current weights\cite{wrr-kblinux}, and they are reset every time the weights are updated.
\paragraph{Load balancer scaling and placement}
For the load balancer location we evaluate the two most extreme scenarios: A single centralized load balancing instance which serves all client requests, and a maximally distributed scenario in which every single node in the system also hosts a load balancer instance.

This gives us four load balancer configurations:
\begin{itemize}
    \item Round Robin centralized
    \item Round Robin on all nodes
    \item Least Response Time centralized
    \item Least response time on all nodes
\end{itemize}

Each of these four configurations is evaluated in three different scenarios, which represent clusters of different size and network topology.
These topologies are oriented on a smart city/urban sensing type application.
\begin{itemize}
    \item City
    \item Nation
    \item Global
\end{itemize}
\paragraph{City}
In this scenario the cluster is set to a size of 100 nodes, which are assumed to all be located in the same city, meaning network latencies between nodes are small.
The city features a data center which consists of about 50\% of the total node count, and features nodes with high compute capability, partially with GPU acceleration.
The rest of the nodes are assumed to be distributed across the city closer to clients, and consist to two thirds of medium performance nodes, and one third low performance nodes.
\paragraph{Nation}
This scenario features a larger cluster that spans over three cities.
Each of those cities features the same relative node distribution as the previous scenario.
We chose the USA as our example, as using a small country such as Austria would not provide significant enough latency differences between cities.
Our scenario features the cities of Chicago with 100 nodes, Seattle with 100 nodes, and New York with 150 nodes.
While nodes within the same city feature extremely low network latency, nodes across two cities have more significant network distances between them.

 \begin{table}[]
\centering
\begin{tabular}{r|ccc}
\multicolumn{1}{c|}{} & \textbf{Chicago} & \textbf{New York} & \textbf{Seattle} \\ \hline
\textbf{Chicago}      & -                & 31ms              & 55ms             \\
\textbf{New York}     & 31ms             & -                 & 75ms             \\
\textbf{Seatte}       & 55ms             & 75ms              & -                \\ \hline
\end{tabular}
\caption{Network latencies between cities in the initial \textbf{nation} evaluation scenario. Latencies are taken from Wonder Network's global ping statistics\cite{wondernetworkGlobalPingStatistics}}
\label{tab:initial_nation_pings}
\end{table}

\begin{table}[]
\centering
\begin{tabular}{r|ccc}
\multicolumn{1}{c|}{} & \textbf{New York} & \textbf{London} & \textbf{Sydney} \\ \hline
\textbf{New York}     & -                 & 86ms            & 204ms           \\
\textbf{London}       & 86ms              & -               & 253ms           \\
\textbf{Sydney}       & 204ms             & 253ms           & -               \\ \hline
\end{tabular}

\caption{Network latencies between cities in the initial \textbf{global} evaluation scenario. Latencies are taken from Wonder Network's global ping statistics\cite{wondernetworkGlobalPingStatistics}}
\label{tab:initial_global_pings}
\end{table}   

The network latencies between the cities are taken from Wonder Network\cite{wondernetworkGlobalPingStatistics}, and can be seen in Table \ref{tab:initial_nation_pings}

\paragraph{Global}
The global scenario once again features three cities, but they are distributed not just within a single country, but across the globe.
The cities are New york with 100 nodes, London with 100 nodes, and Sydney with 150 nodes.
Network latencies can be seen in Table \ref{tab:initial_global_pings}.

\paragraph{Clients}
Each scenarios features a client ratio of 0.6, meaning there are 60\% as many clients as there are compute nodes.
Clients are assumed to be on the edge of the network and thus closest to the medium and small sized compute nodes.

\paragraph{Functions and request load}
We used our basic three function deployments for these experiments.
To isolate the system behaviour on the effect of load balancer implementation, scale and placement we disabled the normal function scaling behaviour.
Instead the system immediately sets a fixed scale for each function such that for each node in the entire cluster a function replica is started.
We consider the functions to be of equal importance, thus each function starts up $\frac{n}{3}$ replicas where $n$ is the total number of nodes.
Lastly all experiments simulate a timeframe of 1000 seconds, and feature a request load of 75\gls{rps}.

\subsubsection{Results}

\begin{table}[]
\begin{tabular}{lrrrr}
\hline
                                 & \multicolumn{4}{c}{mean}                                                                                                                              \\
\textbf{Load Balancer Type}      & \multicolumn{1}{r}{\textbf{TRT}} & \multicolumn{1}{r}{\textbf{FET}} & \multicolumn{1}{r}{\textbf{Net CL-LB}} & \multicolumn{1}{r}{\textbf{Net LB-FX}} \\ \hline
\multicolumn{5}{c}{City Scale Evaluation}                                                                                                                                                \\ \hline
Round Robin centralized          & 0.0\%                            & 0.0\%                            & 0.0\%                                  & 0.0\%                                  \\
Round Robin on all nodes         & 13.3\%                           & 0.0\%                            & 35.4\%                                 & 30.3\%                                 \\
Least Response Time centralized  & 32.7\%                           & 47.9\%                           & 15.8\%                                 & 24.6\%                                 \\
Least Response Time on all nodes & 81.3\%                           & 109.3\%                          & 51.6\%                                 & 61.9\%                                 \\ \hline
\multicolumn{5}{c}{Nation Scale Evaluation}                                                                                                                                              \\ \hline
Round Robin centralized          & 0.0\%                            & 0.0\%                            & 0.0\%                                  & 0.0\%                                  \\
Round Robin on all nodes         & 95.7\%                           & -0.4\%                           & 1028.2\%                               & 34.8\%                                 \\
Least Response Time centralized  & 18.1\%                           & 27.8\%                           & 3.0\%                                  & 34.7\%                                 \\
Least Response Time on all nodes & 312.0\%                          & 86.9\%                           & 1065.6\%                               & 257.5\%                                \\ \hline
\multicolumn{5}{c}{Global Scale Evaluation}                                                                                                                                              \\ \hline
Round Robin centralized          & 0.0\%                            & 0.0\%                            & 0.0\%                                  & 0.0\%                                  \\
Round Robin on all nodes         & 82.8\%                           & -0.7\%                           & 2910.6\%                               & -5.3\%                                 \\
Least Response Time centralized  & 21.0\%                           & 18.4\%                           & 0.0\%                                  & 60.8\%                                 \\
Least Response Time on all nodes & 606.9\%                          & 77.3\%                           & 2997.8\%                               & 428.2\%                                \\ \hline
\end{tabular}
\caption{Mean values of a single experimental run of the initial evaluation in different scenarios. Displayed values are in order: Total response time, function execution time, network time between client and load balancer, and network time between load balancer and function replica}
\label{tab:initial_eval_results_mean}
\end{table}

\begin{table}[]
\begin{tabular}{lrrrr}
\hline
                                 & \multicolumn{4}{c}{50th percentile (median)}                                                                                                          \\
\textbf{Load Balancer Type}      & \multicolumn{1}{r}{\textbf{TRT}} & \multicolumn{1}{r}{\textbf{FET}} & \multicolumn{1}{r}{\textbf{Net CL-LB}} & \multicolumn{1}{r}{\textbf{Net LB-FX}} \\ \hline
\multicolumn{5}{c}{City Scale Evaluation}                                                                                                                                                \\ \hline
Round Robin centralized          & 0.0\%                            & 0.0\%                            & 0.0\%                                  & 0.0\%                                  \\
Round Robin on all nodes         & 26.5\%                           & 0.0\%                            & 34.0\%                                 & 25.1\%                                 \\
Least Response Time centralized  & 34.8\%                           & 183.0\%                          & 5.0\%                                  & 13.5\%                                 \\
Least Response Time on all nodes & 93.3\%                           & 185.6\%                          & 37.0\%                                 & 37.7\%                                 \\ \hline
\multicolumn{5}{c}{Nation Scale Evaluation}                                                                                                                                              \\ \hline
Round Robin centralized          & 0.0\%                            & 0.0\%                            & 0.0\%                                  & 0.0\%                                  \\
Round Robin on all nodes         & 109.0\%                          & -7.9\%                           & 1674.1\%                               & 76.2\%                                 \\
Least Response Time centralized  & 23.5\%                           & 16.1\%                           & 4.3\%                                  & 118.2\%                                \\
Least Response Time on all nodes & 350.4\%                          & 22.9\%                           & 1674.1\%                               & 1316.4\%                               \\ \hline
\multicolumn{5}{c}{Global Scale Evaluation}                                                                                                                                              \\ \hline
Round Robin centralized          & 0.0\%                            & 0.0\%                            & 0.0\%                                  & 0.0\%                                  \\
Round Robin on all nodes         & 27.6\%                           & 0.0\%                            & 5338.8\%                               & 95.0\%                                 \\
Least Response Time centralized  & 9.8\%                            & 0.6\%                            & 0.0\%                                  & 1498.5\%                               \\
Least Response Time on all nodes & 1653.3\%                         & 25.9\%                           & 5338.8\%                               & 4541.5\%                               \\ \hline
\end{tabular}
\caption{50th percentile (i.e. median) values of a single experimental run of the initial evaluation in different scenarios. Displayed values are in order: Total response time, function execution time, network time between client and load balancer, and network time between load balancer and function replica}
\label{tab:initial_eval_results_q50}
\end{table}

Since the experiments feature a significant degree of random sampling in their simulation, we ran each experiment 10 times.
The results presented here are those of a single experimental run, since no runs showed significantly different results.
Table \ref{tab:initial_eval_results_mean} shows the percentage improvement different load balancer types and scales give when compared to the default centralized round robin.
Here the mean values of the \gls{trt}, \gls{fet}, and network times between client and load balancer, and load balancer and function replica are shown.
Both a more sophisticated approach to load balancing and moving load balancer instances closer to the clients shows significant performance improvements, but across the board only the combination of these two improvements (i.e. least response time on all nodes) gives the most significant performance uplift.
We can also see that least response time load balancing not only improves network times between the load balancer and the function replica, but also decreases \gls{fet}.
In addition we observe that the performance improvement given through distributed least response time load balancing becomes larger in geographically more distributed scenarios.

Table \ref{tab:initial_eval_results_q50} shows the difference between the median of these metrics.
Here the performance uplift achieved by least response time load balancing on all nodes become even larger, up to a 16,5x improvement in \gls{trt} in the global scenario.

For even more detailed analyses figure \ref{fig:initial_eval_pdfs} shows the probability density function estimated of the same experimental run.
Note that \textit{E2E RT} in this figure means \gls{trt}.
Here the overall trend observed in the tables \ref{tab:initial_eval_results_mean} and \ref{tab:initial_eval_results_q50} carries through, showing that least response time on all nodes not only improves average performance, but pushes the whole distribution towards faster \gls{fet} and lower network times.

\section{Load Balancer Implementation and Parametrization}
Next, we explore the effect the load balancer implementation and parametrization have.
Because the edge computing environment features different conditions than the cloud, we evaluate whether or not different implementations of weighted round robin affect the system differently.
In addition we evaluate the parameter configuration for our load balancing approach, testing if there are certain configurations that perform better than others and trying to see if there are patterns in the parameters' influence on performance.
\subsection{Least Response Time Load Balancing Implementation}
% got 2,5 pages
With these experiments the goal is to better understand how the implementation details of the load balancing solution affect overall performance.
The previously performed initial evaluation revealed some potentially counter-intuitive behaviour, so to better inform our proposed implementation these experiments evaluate different implementation patterns.

An implementation of least response time load balancing is split in two parts: The gathering of response time data, and the conversion of that data into load balancing decisions.
As we discussed in our approach, we convert the response time data into weights, which are used as inputs to a weighted round robin load balancing implementation that ultimately makes load balancing decisions.
\subsubsection{Setup}
With these experiments our focus lies on these implementations of weighted round robin.
In particular we evaluate four different implementations of weighted round robin:
\begin{itemize}
    \item \textbf{Random:} this implementation uses a simple weighted random distribution. It is used as a baseline to compare other implementations to
    \item \textbf{Classic:} this represents the implementation used in the initial evaluation experiment\cite{wrr-kblinux}. It is deterministic and works through upstreams starting with the ones with the highest weight.
    \item \textbf{Adapted Classic:} an adaptation from the classic implementation, which allows for weights to be updated on the fly without the algorithm having to be restarted.
    \item \textbf{Smooth:} the implementation described in our approach and also used by nginx\cite{nginx-wrr}. It too is deterministic but alternates between upstreams with high and low weights in load balancing decisions.
\end{itemize}

We simulate a least response time load balancer with each of these implementations with 500 upstreams, which are simplified to sample response times from a lognormal distribution, over a timeframe of 500 seconds with a request rate of 5\gls{rps}.
Weights are updated every 15 seconds, except for the smooth weighted round robin implementation which is updated every second, since this is a core benefit afforded by the implementation we want to test explicitly.
The weight range for each implementation is [0;10] and response times get mapped linearly, i.e. using a scaling factor of 1.

\subsubsection{Results}

\begin{figure}
    \centering
    \includegraphics[width=11cm]{graphics/graphs/lb_imp_upstream_coverage.png}
    \caption{A graph showing how quickly each upstream receives at least one request with different weighted round robin implementations for least response time load balancing}
    \label{fig:lb_imp_upstream_coverage}
\end{figure}

\begin{figure}
    \centering
    \includegraphics[width=11cm]{graphics/graphs/lb_imp_trt_convergence.png}
    \caption{Graph showing average response times converging for different weighted round robin implementations with least response time load balancing}
    \label{fig:lb_imp_rt_convergence}
\end{figure}

In our results we can see significant differences between the implementations.
First, as can be seen in Figure \ref{fig:lb_imp_upstream_coverage}, there are significant differences on how fast, or whether at all, every node in the system receives at least one request.
Since least response time load balancing relies on requests to evaluate the performance of an upstream, the quality of the solution is limited by the amount of information available about the upstreams.
We can see that the classic implementation never sends requests to more than a small subset of upstreams, which is due to its internal state being reset on weight updates.
While the random sampling eventually converges toward covering all upstreams, both the smooth and adapted classic implementation do so much more quickly.
The adapted classic is even a slight bit faster than the smooth implementation, but from our point of view we consider this difference to be negligible.

We also observe drastically different behaviour of the average response times between the different implementations.
Figure \ref{fig:lb_imp_rt_convergence} shows that Classic never reaches the level of performance of the other implementations, most likely due to it never discovering the faster subset of upstreams.
While we can also observe that for both the smooth and the random reference implementation average response times stabilize eventually, the adapted classic shows an alternating pattern of fast and slow response time averages.
This is due to the implementation, which works through the highest weighted upstreams first, before including the next highest weighted ones and so on.
Since the order in which this happens is deterministic this oscillating performance pattern forms.
The classic implementation shows the same behaviour, but due to it only ever sending requests to a smaller set of upstreams the pattern isn't as easily visible.
Note that as we are considering response time averages low values on the y-axis in Figure \ref{fig:lb_imp_rt_convergence} are desirable.












% for discussion: shows that default implementations don't necessarily fit least response time / edge computing and need to be adapted. Also shows how unintuitive behaviour can be
% can discuss potential performance tradeoffs between spending "resources" on node discovery vs. using already known nodes that work well.














\section{Load Balancer Parametrization}
% with graphics 2 pages

\section{Resource Usage and Load Balancer Scale}
After load balancer implementation and parametrization have been evaluated, we now move on to load balancer resource usage and scale.
The initial evaluation already shows that having distributed load balancers improves system performance.
With larger numbers of load balancers also come increased resource costs, however.
To find out the exact resource costs we measure the system resource usage of our load balancing approach on different physical hardware.
The second factor influencing the overall cost of distributed load balancers is the number of load balancers deployed.
Intuitively a greater number of load balancers will result in better overall performance, as load balancers will then have a better chance at being close to clients and function replicas.
In the second evaluation we explore the relationship between load balancer scale and system performance.
Together these two evaluations give us the insight needed to make an informed trade-off between total resource consumption and system performance.
\section{Load Balancer Resource Usage}
\subsection{Resource Evaluation Setup}
% 2 pages -> got 4
The goal of these experiments is to understand the resource consumption one can expect from a load balancer in the context of serverless edge computing.
Using the traces of these experiments we implement the load balancers within the serverless simulator as close to reality as possible, thus making sure the simulation results are more representative.

As described we use a Kubernetes cluster with a number of different nodes to make real-life measurements of load balancer's performance characteristics, requests are sent via galileo\cite{galileo-github}\cite{operating-energy-aware-galileo}, and the system metrics are measured via telemd\cite{telemd-github}.
Since the load balancer is containerized, just like it would be in a serverless framework, telemd relies on the metrics provided by Kubernetes to measure resource consumption.
Specifically, telemd reports the values provided by the following files at a set interval\cite{telemd-github}.
\begin{lstlisting}[language=Bash]
/sys/fs/cgroup/cpuacct/kubepods/besteffort/pod<pod ID>/<container ID>/cpuacct.usage
/sys/fs/cgroup/memory/kubepods/besteffort/pod<pod ID>/<container ID>/memory.stat
\end{lstlisting}
Network characteristics are not taken into account with these measurements, as they are highly application specific and experimental evaluation is thus neither fruitful nor necessary.

Since scaling and scheduling are important aspects of our approach, we also consider the image size of the load balancer to be relevant.
Because Docker Hub is one of the largest hosting platforms for application container images, we use the information it provides\cite{traefik-dockerhub} to determine the image size for different platforms.

To cover a wide range of possible deployment scenarios the nodes we use for this experiment cover hardware from powerful desktop processors as they are used in large scale data centers to small scale reduced instruction set processors as one finds in mobile devices.

\begin{table}[]
\begin{tabular}{lllll}
\hline
\textbf{Node} & \textbf{\begin{tabular}[c]{@{}l@{}}Processor\\ Architecture\end{tabular}} & \textbf{\begin{tabular}[c]{@{}l@{}}logical CPU\\ core count\end{tabular}} & \textbf{\begin{tabular}[c]{@{}l@{}}RAM\\ (GiB)\end{tabular}} & \textbf{Model Name}           \\ \hline
AMD           & x86/amd64                                                                 & 8                                                                         & 32                                                           & AMD Ryzen Embedded V1605B     \\
Intel         & x86/amd64                                                                 & 8                                                                         & 16                                                           & Intel Xeon E3-1230 v6         \\
Jetson        & aarch64                                                                   & 4                                                                         & 4                                                            & NVIDIA Jetson Nano            \\
RockPi        & aarch64                                                                   & 6                                                                         & 4                                                            & RockPi 4B                     \\
RPi           & ARMv7                                                                     & 4                                                                         & 1                                                            & Raspberry Pi 4Model B Rev 1.1
\end{tabular}
\caption{Nodes present in the real Kubernetes cluster used for resource consumption evaluation}
\label{tab:k8s_nodes}
\end{table}



Table \ref{tab:k8s_nodes} shows the different nodes that are present in the cluster and which get evaluated. Aside from covering typical compute resources we also include a number of ARM devices, one with special purpose compute acceleration, to show how mobile edge computing devices or devices built for purposes such as image processing would perform.

To evaluate load balancer's resource consumption we need to select a real load balancer implementation that serves as a stand-in for generic load balancers.
For this purpose we selected traefik proxy\cite{traefik}, a level 7 load balancer and application proxy.
We considered it a fitting example of a real world load balancer since it is available on all the different platforms we want to test, supports the kind of complex load balancing required of our proposed approach.
It is also open source, which allowed us to extend it with least response time load balancing capabilities like we proposed, which further moves these experiments closer to real-world conditions.
This modified version\cite{traefik-jjnp} is also open source and is already functional, although it does not feature the integrations that would allow its use in a serverless framework.

Since in the experimental setup requires actual requests being sent we also need an upstream the load balancer can then forward the requests it receives to.
For this experiment the upstream is a separate application that does nothing but respond to the requests with a given payload.
Its only distinguishing feature is that we can choose from a random distribution that determines how long the application waits before responding, thus simulating the \gls{fet} usually experienced within a serverless system.
For the general evaluation of different devices we simulated a response time/\gls{fet} of 20 milliseconds.
We built this application specifically for this purpose, and also made it publicly available\cite{palecekResponder2021} so that it can be used by others.
We oriented our payload on a hypothetical edge intelligence application, where a client sends an image for it to then be classified.
Our payload is thus a compressed JPEG image with a file size of 250KiB, while the response is a simple JSON with negligible size.

Lastly for the request load we decided to test a range of different loads.
To do this in a simple and presentable way we send requests along a pattern, which steadily increases the amount of requests sent from 0 to 250 requests per second over a period of 300 seconds, then continuously sending 250 requests per second for 60 more seconds before stopping completely.

Apart from the evaluation of different types of devices we performed another test with the same request pattern, but one time with the previously used response time of 20ms and once again with a response time of 250ms to see whether the response time of the upstreams makes a difference to the load balancer resource consumption.
\subsection{Observed Resource Consumption}
Our results show that the resource consumption of load balancers of this class is, generally speaking, quite moderate.
At a rate of 250 requests per seconds, a substantial load, CPU utilization ranges from 4,3\% to 34\%, while memory consumption is between 12 and 59MiB.
The results at 250 requests per second can be seen in Table \ref{tab:resource_eval_results}.

The relationship between the resource consumption and request load can be seen in Figure \ref{fig:lb_resources_by_type}.
From these graphs we can see that CPU utilization rises almost linearly with the number of concurrent requests, while the RAM consumption does not show a comparable pattern.
In fact, the memory consumption shows significant variance between the different nodes with some nodes showing more than quadruple the system memory usage of others.
There is also no obvious explanatory pattern to be observed, as these wide ranges of memory consumption also exist within nodes of the same processor architecture.

The response time of the upstream services also appears to influence resource utilization as can be seen in Figure \ref{fig:lb_resources_by_rt}.
While the CPU utilization seems to be consistent throughout the experiments with 20ms and 250ms response time, the memory consumption is higher for the case of 250ms.

Lastly, the size of the different Docker images, which can be seen in Table \ref{tab:resource_eval_results}, does differ across processor architectures, but not to a significant degree.
The difference between the largest and the smallest images is 2,28Mib, or 8\% to 8,5\%, depending on which image is considered the default basis for the calculation.


\begin{table}[]
\begin{tabular}{lrrr}
\hline
\textbf{Node} & \textbf{CPU utilization} & \textbf{RAM Usage} & \textbf{Docker Image Size} \\ \hline
AMD           & 4,3\%                    & 57 MiB             & 28.46 MB                   \\
Intel         & 6,5\%                    & 16 MiB             & 28.46 MB                   \\
Jetson Nano   & 12,5\%                   & 16 MiB             & 26.18 MB                   \\
RockPi        & 13\%                     & 59 MiB             & 26.18 MB                   \\
RPi           & 34\%                     & 12 MiB             & 26.75 MB                  
\end{tabular}
\caption{Results of the load balancer resource evaluation at 250 requests per second}
\label{tab:resource_eval_results}
\end{table}
\begin{figure}
    \centering
    \includegraphics[width=14cm]{graphics/graphs/lb_resources_by_device.png}
    \caption{Resource consumption of the traefik\cite{traefik} load balancer on different devices with a response time of 20ms and request payload size of 250KiB}
    \label{fig:lb_resources_by_type}
\end{figure}

\begin{figure}
    \centering
    \includegraphics[width=11cm]{graphics/graphs/lb_resources_by_response_time.png}
    \caption{Resource consumption of the traefik\cite{traefik} load balancer with different response times on and a 250KiB request payload}
    \label{fig:lb_resources_by_rt}
\end{figure}

\input{chapters/6_Evaluation/5_lb_scale}


\section{Osmotic Scaling and Scheduling}

Finally, we evaluate our osmotic scaling and scheduling approach.
Our evaluations pursue a number of goals.
First, we evaluate how our scaling and scheduling approach affects the serverless system, both in terms of performance and in regard to its key metrics.
Of particular interest is how our approach affects the scaling decisions of functions when simulating the default scaling behaviour of serverless frameworks like OpenFaaS\cite{openfaas}.
Second, we explore the relationship between the osmotic pressure threshold and the system performance.
The threshold affects the load balancer scale, which in turn affects the system performance as previous evaluations showed.
This makes it a key parameter to determine both the performance the system provides and the amount of resources it consumes to run its load balancers.
Third and last, we evaluate the behaviour of our approach under dynamic system conditions, which are a key facet of edge computing.
In particular we examine how the system behaves when the location in the network requests originate from changes over time.

\subsection{Performance of Osmotic Scaling and Load Balancing}
% 2 pages
With this experiment we want to provide baseline performance data for the osmotic scaling and scheduling method we propose. The goal is to show how our proposed solution operates without fine-tuning of parameters, or any other conditions. The experiment should also show how the osmotic scaling and scheduling of load balancers affects other parts of the serverless system, most notably the scaling decisions of regular functions.
\subsubsection{Setup}
Our experiment setup for these evaluations is once again based on the serverless edge computing simulator we also used for the initial evaluations. To stay consistent we used the same network topologies from the previous experiment that investigated the impact of load balancer scale on the system, meaning we assume clients to typically be connected via a mobile network, and compute resources to be distributed on the edge. The three topologies we tested are once again one scenario with a single city, one with three cities on the same continent, and one with three cities distributed across the globe. The cities chosen, along with the network latencies between them are the same as in the initial evaluation namely Chicago, New York, and Seattle for the nation-distibuted and New York, London, and Sydney for the globally-distributed experiment. The network latencies between them can be seen in tables \ref{tab:initial_nation_pings} and \ref{tab:initial_global_pings} respectively.

To stay consistent with the other experiments, and partially due to performance limitations, we once again tested each topology scenario with 25\gls{rps}, 50\gls{rps}, and 75\gls{rps}. As for the osmotic scaling and scheduling component, we set the pressure threshold for scaling up to 0.025 and the downscaling threshold to 0.03, which can roughly be read as the system requiring an expected \gls{trt} improvement of 2.5\% and 3\% to add or remove a load balancer instance on a given node. Bear in mind that this idea of required estimated performance improvement is a mental model to get a more intuitive understanding for the parameters, and is not equivalent to the actual implementation.

The last way in which the experiment setup differs from the previous experiments is that there is a function scheduling component active. While the other experiments purposefully set a fixed scale for each of the serverless functions in the system to avoid it as a confounding variable, these experiments use a dynamic function scaler to show how this type of load balancer scaling and scheduling affects the overall system.
In concrete terms, we set the simulator up to use a set rate of average requests per function replica.
The reasoning behind this choice is that OpenFaaS uses the same methodology as a default configuration, which we use as a stand-in example of serverless computing frameworks in general.
For the osmotic scaling and scheduling parameters we used 0.03 as a scale-up threshold and 0.05 as a scale-down threshold.

\subsubsection{Results}

\begin{table}[]
\begin{tabular}{lrrrrrr}
\hline
\textbf{Experiment}  & \textbf{\begin{tabular}[c]{@{}r@{}}LB\\ replicas\end{tabular}} & \textbf{\begin{tabular}[c]{@{}r@{}}Converged\\ Total \\ Function\\ Replicas\end{tabular}} & \textbf{\begin{tabular}[c]{@{}r@{}}Cross-City\\ Request\\ Share\end{tabular}} & \textbf{\begin{tabular}[c]{@{}r@{}}Mean\\ TRT\end{tabular}} & \textbf{\begin{tabular}[c]{@{}r@{}}Median\\ TRT\end{tabular}} & \textbf{\begin{tabular}[c]{@{}r@{}}Q90\\ TRT\end{tabular}} \\ \hline
25rps City LRT       & 6                                                              & 88                                                                                        & 0.0\%                                                                         & 121ms                                                       & 121ms                                                         & 157ms                                                      \\
25rps City Osmotic   & 24                                                             & 89                                                                                        & 0.0\%                                                                         & 117ms                                                       & 119ms                                                         & 149ms                                                      \\
25rps City RR        & 6                                                              & 90                                                                                        & 0.0\%                                                                         & 168ms                                                       & 136ms                                                         & 275ms                                                      \\ \hline
25rps Nation LRT     & 14                                                             & 89                                                                                        & 17.0\%                                                                        & 154ms                                                       & 131ms                                                         & 254ms                                                      \\
25rps Nation Osmotic & 3                                                              & 88                                                                                        & 32.5\%                                                                        & 224ms                                                       & 208ms                                                         & 386ms                                                      \\
25rps Nation RR      & 14                                                             & 90                                                                                        & 65.4\%                                                                        & 274ms                                                       & 266ms                                                         & 432ms                                                      \\ \hline
25rps Global LRT     & 14                                                             & 94                                                                                        & 0.4\%                                                                         & 142ms                                                       & 126ms                                                         & 210ms                                                      \\
25rps Global Osmotic & 5                                                              & 90                                                                                        & 1.7\%                                                                         & 152ms                                                       & 128ms                                                         & 224ms                                                      \\
25rps Global RR      & 14                                                             & 99                                                                                        & 65.4\%                                                                        & 501ms                                                       & 410ms                                                         & 937ms                                                      \\ \hline
50rps City LRT       & 6                                                              & 249                                                                                       & 0.0\%                                                                         & 127ms                                                       & 124ms                                                         & 177ms                                                      \\
50rps City Osmotic   & 13                                                             & 249                                                                                       & 0.0\%                                                                         & 123ms                                                       & 123ms                                                         & 164ms                                                      \\
50rps City RR        & 6                                                              & 249                                                                                       & 0.0\%                                                                         & 162ms                                                       & 141ms                                                         & 248ms                                                      \\ \hline
50rps Nation LRT     & 14                                                             & 179                                                                                       & 17.7\%                                                                        & 153ms                                                       & 129ms                                                         & 270ms                                                      \\
50rps Nation Osmotic & 5                                                              & 177                                                                                       & 24.3\%                                                                        & 165ms                                                       & 135ms                                                         & 283ms                                                      \\
50rps Nation RR      & 14                                                             & 181                                                                                       & 65.0\%                                                                        & 271ms                                                       & 264ms                                                         & 430ms                                                      \\ \hline
50rps Global LRT     & 14                                                             & 190                                                                                       & 1.2\%                                                                         & 147ms                                                       & 128ms                                                         & 213ms                                                      \\
50rps Global Osmotic & 4                                                              & 181                                                                                       & 3.8\%                                                                         & 159ms                                                       & 128ms                                                         & 238ms                                                      \\
50rps Global RR      & 14                                                             & 197                                                                                       & 65.1\%                                                                        & 494ms                                                       & 410ms                                                         & 922ms                                                      \\ \hline
75rps City LRT       & 6                                                              & 249                                                                                       & 0.0\%                                                                         & 127ms                                                       & 123ms                                                         & 175ms                                                      \\
75rps City Osmotic   & 14                                                             & 249                                                                                       & 0.0\%                                                                         & 122ms                                                       & 122ms                                                         & 162ms                                                      \\
75rps City RR        & 6                                                              & 249                                                                                       & 0.0\%                                                                         & 168ms                                                       & 140ms                                                         & 275ms                                                      \\ \hline
75rps Nation LRT     & 14                                                             & 269                                                                                       & 14.9\%                                                                        & 150ms                                                       & 129ms                                                         & 260ms                                                      \\
75rps Nation Osmotic & 5                                                              & 264                                                                                       & 24.1\%                                                                        & 169ms                                                       & 138ms                                                         & 287ms                                                      \\
75rps Nation RR      & 14                                                             & 269                                                                                       & 65.2\%                                                                        & 273ms                                                       & 269ms                                                         & 433ms                                                      \\ \hline
75rps Global LRT     & 14                                                             & 278                                                                                       & 0.1\%                                                                         & 141ms                                                       & 128ms                                                         & 214ms                                                      \\
75rps Global Osmotic & 5                                                              & 268                                                                                       & 1.9\%                                                                         & 152ms                                                       & 129ms                                                         & 223ms                                                      \\
75rps Global RR      & 14                                                             & 285                                                                                       & 65.2\%                                                                        & 497ms                                                       & 406ms                                                         & 928ms                                                      \\ \hline
\end{tabular}
\caption{Osmotic baseline evaluation results}
\label{tab:osmotic_base}
\end{table}

\begin{figure}
    \centering
    \includegraphics[width=12cm]{graphics/graphs/osmotic_base_function_scale_by_lb.png}
    \caption{Total scale of all functions for each load balancer scaling/schedulilng method}
    \label{fig:osmotic_fx_scale_by_scaling}
\end{figure}

Table \ref{tab:osmotic_base} shows the results of the experiment.
In terms of mean \gls{trt} performance the results of our proposed osmotic scaling and scheduling method are similar to those of the fixed scaling LRT reference setup, albeit between 1\% and 45\% worse with most scenarios only between 2\% and 8\% worse.
Differences between least response time static, and osmotic scaling are less pronounced in the median and 90th percentile \glspl{trt} as can also be seen in Table \ref{tab:osmotic_base}.
The statically scaled round robin load balancing is, as one would expect, significantly worse.
It does, however, give a good impression of the performance improvements possible based on our approach in a more complex and realistic deployment scenario than the initial evaluation.

We can also observe that in the city scenario the osmotic scaling and scheduling ends up deploying the more load balancer instances than the static scaling, but that for the nation- and globally-distributed network topology scenarios this patterns is reversed. There the osmotic scaler deploys only about a third of the replicas of the static scaling.

A similar pattern can be seen with regard to function scaling.
The osmotic scaling leads to between 1\% and 6\% fewer function replicas being deployed. Figure \ref{fig:osmotic_fx_scale_by_scaling} shows how the timing and frequency of scaling decisions are not different between the scaling methods, but osmotic scaling results in slightly fewer function replicas. 
\subsection{Optimization Aggressiveness with Osmotic Scaling}
We already learned from the experiment about load balancer scale that there is a tendency for a higher scale of load balancers to ultimately result in better performance.
With this experiment we want to test the interplay of this phenomenon with the osmotic load balancing and scaling we propose. With our osmotic approach we can set the scale-up threshold to more or less arbitrary values to influence how quickly or slowly the system scales up the number of load balancers, and how many load balancers will thus ultimately end up being deployed.

\subsubsection{Setup}
To test the performance we simulate a number of different configuration scenarios. For the rest of the system environment we choose to reuse the globally distributed topology from previous experiments with a request rate of 75\gls{rps} being simulated over the course of 2000 seconds.

For the parameters of the osmotic scaling and scheduling we run experiments for a range of scale-up thresholds ranging from 0.02 to 0.1.
The scale down threshold is always set to 0.2, which is relatively high, because we explicitly want to test how the scale-up threshold can be used to determine how many load balancers the system will deploy to optimize response times.

\subsubsection{Results}

\begin{table}[]
\begin{tabular}{lrrrrr}
\hline
\textbf{\begin{tabular}[c]{@{}l@{}}Upscaling\\ Pressure\\ Threshold\end{tabular}} & \textbf{\begin{tabular}[c]{@{}r@{}}LB\\ Replicas\end{tabular}} & \textbf{\begin{tabular}[c]{@{}r@{}}Mean\\ TRT\end{tabular}} & \textbf{\begin{tabular}[c]{@{}r@{}}Mean\\ FET\end{tabular}} & \textbf{\begin{tabular}[c]{@{}r@{}}Mean\\ LB\_FX\end{tabular}} & \textbf{\begin{tabular}[c]{@{}r@{}}Mean\\ CL\_LB\end{tabular}} \\ \hline
0.1                                                                               & 4                                                              & 155.8ms                                                     & 29.1ms                                                      & 33.5ms                                                         & 92.7ms                                                         \\
0.09                                                                              & 4                                                              & 152.5ms                                                     & 29.7ms                                                      & 30.1ms                                                         & 92.2ms                                                         \\
0.08                                                                              & 6                                                              & 152.1ms                                                     & 29.5ms                                                      & 30.0ms                                                         & 92.2ms                                                         \\
0.07                                                                              & 5                                                              & 152.5ms                                                     & 29.3ms                                                      & 30.4ms                                                         & 92.4ms                                                         \\
0.06                                                                              & 5                                                              & 148.5ms                                                     & 29.6ms                                                      & 26.6ms                                                         & 91.6ms                                                         \\
0.05                                                                              & 5                                                              & 147.5ms                                                     & 29.9ms                                                      & 25.3ms                                                         & 91.3ms                                                         \\
0.04                                                                              & 5                                                              & 148.5ms                                                     & 29.8ms                                                      & 26.5ms                                                         & 91.5ms                                                         \\
0.03                                                                              & 7                                                              & 136.8ms                                                     & 26.4ms                                                      & 19.6ms                                                         & 91.2ms                                                         \\
0.02                                                                              & 20                                                             & 139.0ms                                                     & 24.8ms                                                      & 25.3ms                                                         & 90.2ms                                                         \\ \hline
\end{tabular}
\caption{Response time performance metrics for different upscaling pressure thresholds}
\label{tab:osmotic_scaling_aggressiveness}
\end{table}





\begin{figure}
    \centering
    \includegraphics[width=12cm]{graphics/graphs/osmotic_optim_thres_vs_trt.png}
    \caption{\gls{trt} compared to current load balancer scale for two different osmotic scaling parametrizations}
    \label{fig:osmotic_trt_vs_replica_scale}
\end{figure}

The results show two clear trends.
First that will lower upscaling pressure thresholds more load balancer replicas are being deployed by the osmotic scaling component, and second that the mean response time improves with higher number of load balancer replicas.
As Table \ref{tab:osmotic_scaling_aggressiveness} shows, there is a diminishing return with higher numbers of load balancers, at least in the topology scenario tested.
The results also show that while higher numbers of load balancers provide better performance once the load balancers have gathered enough information about available replicas, this process is faster the fewer load balancers there are, thus giving better performance early on.
The behaviour can be observed easily in the graph in Figure \ref{fig:osmotic_trt_vs_replica_scale}.
\subsection{Osmotic Scheduling in Dynamic Systems}
% 3+ pages (complicated setup, no?)
In this last experiment we test the behaviour of our osmotic scaling and scheduling method in a dynamic system.
Since dynamic changes of the system make-up are a core part of edge computing, and our approach is explicitly constructed with these dynamic factors in mind, we believe it is important to test the efficacy of our approach in such a scenario.
Because a lot of components have already been analyzed in-depth, and results are most clear when only one factor is tested at any given time, we choose to use the request origin as the dynamic  system component.
With the experiment we test how our proposed approach handles requests originating from different regions of the overall system over time.
\subsubsection{Setup}
For this experiment we once again use the globally distributed scenario as our network topology, and apply a constant request rate of 25\gls{rps}.
Each of the three cities present in the topology additionally has a probability function associated with it, which determines the chance of a request originating from that city.
These probability functions for the cities are set up in such a way that most of the requests originate from only one of the cities for a given time period.
After some time the active city changes and the requests gradually start to come from another city.
The periods and changes are set up in such a way that over the course of the 2000 second long simulation each of the cities is the main originator of requests at one point.

\subsubsection{Results}

\begin{figure}
    \centering
    \includegraphics[width=14cm]{graphics/graphs/osmotic_dynamic_region_rps_lb_relicas.png}
    \caption{Load balancer replica count per city over changing request origins}
    \label{fig:osmotic_dynamic_lb_replicas}
\end{figure}

\begin{figure}
    \centering
    \includegraphics[width=14cm]{graphics/graphs/osmotic_dynamc_region_rps_trt.png}
    \caption{Total response time over changing request origins}
    \label{fig:osmotic_dynamic_trt}
\end{figure}

\begin{figure}
    \centering
    \includegraphics[width=14cm]{graphics/graphs/osmotic_dynamic_region_tx_times.png}
    \caption{Client to load balancer, and load balancer to function transfer times over changing request origins}
    \label{fig:osmotic_dynamic_tx}
\end{figure}

First of all, the results show that the osmotic scaling and scheduling component does indeed take the request origin into account when deciding on the number and location of new load balancer replicas.
Figure \ref{fig:osmotic_dynamic_lb_replicas} shows this in action.
While originally load balancers are only spawned in one city, because all requests originate from it, once requests start coming from another city another load balancer instance is deployed in that city.

The effect this has on the system at large can also be observed easily, as Figure \ref{fig:osmotic_dynamic_trt} shows.
Here we can see that while the total response time of the system spikes once requests start to originate from another city, it starts to stabilize and come down to previous levels again once load balancers are present in the new city.
Please note that the \gls{trt} values shown in Figure \ref{fig:osmotic_dynamic_trt} are a moving average over a 10 second window, since this removes noise from the plot, making it more readable.
Likewise the request rate per city in Figures \ref{fig:osmotic_dynamic_lb_replicas}, \ref{fig:osmotic_dynamic_trt}, and \ref{fig:osmotic_dynamic_tx} is a moving average over a 5 second window, and displays the request rate for all functions deployed in the system.

Just like with the \gls{trt}, Figure \ref{fig:osmotic_dynamic_tx} shows that the request transfer time between client and load balancer, as well as between load balancer and function replica also spike when traffic originates from a different city.
There too, though we see that it returns to previous levels once load balancer replicas become available near the request origin.
