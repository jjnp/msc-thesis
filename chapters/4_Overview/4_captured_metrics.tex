% 1-2 pages
\section{Captured Metrics}
There is a significant number of metrics that could be captured using such a simulation. With FaaS-Sim we capture traces of the requests being sent, and major system events.
The major system events are the addition or removal of deployments, as well as all scaling and scheduling decisions with respect to functions and load balancers.
For requests our traces provide the following information: \gls{trt}, \gls{fet}, waiting time, request client, load balancer instance, function instance, and the network times between client, load balancer and function instance.
The waiting time refers to the elapsed between a request having been received by a function instance, and the request being processed.
Waiting times occur if a node receives requests faster than it can process them.
In terms of resource usage, the resources reserved by the simulated Kubernetes pods, which correspond to the requirements defined in Kubernetes deployment manifests, are also recorded.
Note that this does not necessarily correspond to actual resource usage.
Because these reserved resource metrics are used for Kubernetes' scheduling decisions, we believe they are still worth being captured.

In keeping with the set of potential metrics outlined by Aslanpour, Gill and Toosi\cite{aslanpourPerformanceEvaluationMetrics2020a}, we pay particular attention to response times, potential \gls{sla} levels and oscillation mitigation.
We also introduce more qualitative metrics, related to our particular evaluation scenario, such as the share of requests being routed outside of the city or network region they originate in.